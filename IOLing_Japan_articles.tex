\documentclass[11pt,a4paper,platex,titlepage]{jsarticle}
\usepackage{graphicx}
\usepackage[truedimen,left=25truemm,right=25truemm,top=25truemm,bottom=25truemm]{geometry}
\usepackage{caption}
\usepackage{jpdoc}%https://github.com/ricosjp/Articles/

\usepackage{calc}
\usepackage[center]{titlesec}
\newlength{\wtarget}
\newlength{\wactual}
\newcommand*{\kintou}[2]{\kintouwidth{#1zw}{#2}}
\newcommand*{\kintouwidth}[2]{%
    \setlength{\wtarget}{#1}%
    \settowidth{\wactual}{#2}%
    \ifthenelse{\lengthtest{\wtarget < \wactual}}{%
        \setlength{\wtarget}{1pt * \real{\strip@pt\wtarget} / \real{\strip@pt\wactual}}%
        \scalebox{\strip@pt\wtarget}[1]{#2}%
    }{%
        \makebox[\wtarget][s]{#2}%
    }%
}

\renewcommand{\thesubsection}{第\arabic{subsection}章}
\renewcommand{\thearticle}{第\arabic{Article}条}
\renewcommand{\thesubarticle}{第\arabic{Article}条の\arabic{Branch}}
\renewcommand{\thetermno}{\arabic{TermNo}}
\renewcommand{\theitemi}{(\arabic{enumi})}
\renewcommand{\theitemiii}{\thekansuji{enumiii}}
\makeatletter
\renewcommand\article[1]{%
\refstepcounter{Article}%
\setcounter{TermNo}{1}%
\protected@edef\@currentlabel{\thearticle}%
\setcounter{TermNo}{1}\setcounter{Branch}{1}%
\par\vspace{3mm}\normalfont\normalsize(#1)%
\par\noindent\hangindent=20mm%
\makebox[20mm][l]{\thearticle}%
}
\renewcommand\subarticle[1]{%
\refstepcounter{Branch}%
\setcounter{TermNo}{1}%
\protected@edef\@currentlabel{\thesubarticle}%
\par\vspace{3mm}\normalfont\normalsize(#1)%
\par\noindent\hangindent=20mm%
\makebox[20mm][l]{\thesubarticle}%
}
\makeatother

\title{\fontsize{40truept}{40truept}\selectfont\bf\kintou{5}{定款}}
\author{\bf\Large \kintou{13}{特定非営利活動法人}\medskip\\ \bf\LARGE \kintou{20}{国際言語学オリンピック日本委員会}}
\date{\bf\Large 制定 平成15年10月6日} 

\begin{document}

\pagestyle{empty}

\maketitle

\newpage

{\centering \Large\bf 特定非営利活動法人国際言語学オリンピック日本委員会定款 \vskip 0em}
\vskip 2em

\subsection{総則}
\article{名称}
この法人は、特定非営利活動法人国際言語学オリンピック日本委員会という。また、英文名をInternational Linguistics Olympiad Committee of Japanといい、略称をIOL日本委員会とする。

\article{事務所}
この法人は、主たる事務所を東京都府中市白糸台1丁目53番地の56に置く。

\article{目的}
この法人は、高等学校・中学校等の生徒を対象として、国際言語学オリンピックに出場する代表者選抜を主たる目的とした日本言語学オリンピック大会の開催、国際言語学オリンピックへの日本代表選手の派遣及び研修、国際言語学オリンピックに関する普及啓発事業並びに生徒に対する言語学の理解増進に関する事業を行い、言語学の才能・素質のある生徒を見出すとともに、国際的な研究者・教育関係者の交流を通じ、わが国の言語学の普及啓発に寄与することを目的とする。
\label{目的}

\article{特定非営利活動の種類}
この法人は、前条の目的を達成するため、次の種類の特定非営利活動を行う。
\begin{enumerate}
	\item 学術、文化、芸術又はスポーツの振興を図る活動
	\item 国際協力の活動
\end{enumerate}

\article{事業の種類}
この法人は、\ref{目的}の目的を達成するため、特定非営利活動に係る事業として、次の事業を行う。
\begin{enumerate}
	\item 日本言語学オリンピック大会の開催に関する事業
	\item 国際言語学オリンピックへの日本代表選手の派遣に関する事業
	\item 国際言語学オリンピックへの日本代表選手の研修に関する事業
	\item 国際言語学オリンピックに関する普及啓発事業
	\item 言語学の理解増進を図るための事業
\end{enumerate}

\subsection{会員}
\article{種別}
この法人の会員は、次の2種とし、正会員をもって特定非営利活動促進法(以下「法」という。)上の社員とする。
\begin{enumerate}
	\item \kintou{4}{正会員} この法人の目的に賛同して入会した個人及び団体
	\item 賛助会員 この法人の目的に賛同し賛助するために入会した個人及び団体
\end{enumerate}

\article{入会}
会員の入会について、特に条件は定めない。
\term 会員として入会しようとするものは、理事長が別に定める入会申込書により、理事長に申し込むものとする。
\term 理事長は、前項の申し込みがあったとき、正当な理由がない限り、入会を認めなければならない。
\term 理事長は、第2項のものの入会を認めないときは、速やかに、理由を付した書面をもって本人にその旨を通知しなければならない。

\article{入会金及び会費}
会員は、総会において別に定める入会金及び会費を納入しなければならない。入会金及び会費は返還しない。

\article{会員の資格の喪失}
会員が次の各号の一に該当する場合には、その資格を喪失する。
\begin{enumerate}
	\item 退会届の提出をしたとき。
	\item 本人が死亡し、若しくは失踪宣告を受け、又は会員である団体が消滅したとき。
	\item 継続して2年以上会費を滞納したとき。
	\item 除名されたとき。
\end{enumerate}

\article{退会}
会員は、理事長が別に定める退会届を理事長に提出して、任意に退会することができる。

\article{除名}
会員が次の各号の一に該当する場合には、総会の議決により、これを除名することができる。
\begin{enumerate}
	\item この定款に違反したとき。
	\item この法人の名誉を傷つけ、又は目的に反する行為をしたとき。
\end{enumerate}
\term 前項の規定により会員を除名しようとする場合は、議決の前に当該会員に弁明の機会を与えなければならない。


\subsection{役員}
\article{種別及び定数}
この法人に、次の役員を置く。
\begin{enumerate}
	\item 理事3人以上
	\item 監事1人以上
\end{enumerate}
\term 理事のうち1人を理事長とし、1人を副理事長とする。

\article{選任等}
理事及び監事は、総会において選任する。
\term 理事長及び副理事長は、理事の互選とする。
\term 役員のうちには、それぞれの役員について、その配偶者若しくは3親等以内の親族が1人を超えて含まれ、又は当該役員並びにその配偶者及び3親等以内の親族が役員の総数の3分の1を超えて含まれることになってはならない。
\term 法第20条各号のいずれかに該当する者は、この法人の役員になることができない。
\term 監事は、理事又はこの法人の職員を兼ねてはならない。

\article{職務}
理事長は、この法人を代表し、その業務を総理する。
\term 理事長以外の理事は、法人の業務について、この法人を代表しない。
\term 副理事長は、理事長を補佐し、理事長に事故があるとき又は理事長が欠けたときは、その職務を代行する。
\term 理事は、理事会を構成し、この定款の定め及び総会又は理事会の議決に基づき、この法人の業務を執行する。
\term 監事は、次に掲げる職務を行う。
\begin{enumerate}
	\item 理事の業務執行の状況を監査すること。
	\item この法人の財産の状況を監査すること。
	\item 前2号の規定による監査の結果、この法人の業務又は財産に関し不正の行為又は法令若しくは定款に違反する重大な事実があることを発見した場合には、これを総会又は所轄庁に報告すること。
	\item 前号の報告をするために必要がある場合には、総会を招集すること。
	\item 理事の業務執行の状況又はこの法人の財産の状況について、理事に意見を述べること。
\end{enumerate}

\article{任期等}
役員の任期は、2年とする。ただし、再任を妨げない。
\term 補欠のため、又は増員により就任した役員の任期は、それぞれの前任者又は現任者の任期の残存期間とする。
\term 役員は、辞任又は任期満了後においても、後任者が就任するまでは、その職務を行わなければならない。

\article{欠員補充}
理事又は監事のうち、その定数の3分の1を超える者が欠けたときは、遅滞なくこれを補充しなければならない。

\article{解任}
役員が次の各号の一に該当する場合には、総会の議決により、これを解任することができる。
\begin{enumerate}
	\item 心身の故障のため、職務の遂行に堪えないと認められるとき。
	\item 職務上の義務違反その他役員としてふさわしくない行為があったとき。
\end{enumerate}
\term 前項の規定により役員を解任しようとする場合は、議決の前に当該役員に弁明の
機会を与えなければならない。

\article{報酬等}
役員は、その総数の3分の1以下の範囲内で報酬を受けることができる。
\term 役員には、その職務を執行するために要した費用を弁償することができる。
\term 前2項に関し必要な事項は、総会の議決を経て、理事長が別に定める。


\subsection{会議}
\article{種別}
この法人の会議は、総会及び理事会の2種とする。
\term 総会は、通常総会及び臨時総会とする。

\article{総会の構成}
総会は、正会員をもって構成する。

\article{総会の権能}
総会は、以下の事項について議決する。
\begin{enumerate}
	\item 定款の変更
	\item 解散及び合併
	\item 会員の除名
	\item 事業計画及び予算並びにその変更
	\item 事業報告及び決算
	\item 役員の選任及び解任
	\item 役員の職務及び報酬
	\item 入会金及び会費の額
	\item 資産の管理の方法
	\item 借入金(その事業年度内の収益をもって償還する短期借入金を除く。第47条において同じ。)その他新たな義務の負担及び権利の放棄
	\item 解散における残余財産の帰属
	\item 事務局の組織及び運営
	\item その他運営に関する重要事項
\end{enumerate}

\article{総会の開催}
通常総会は、毎年1回開催する。
\term 臨時総会は、次に掲げる場合に開催する。
\begin{enumerate}
	\item 理事会が必要と認め、招集の請求をしたとき。
	\item 正会員総数の5分の1以上から会議の目的を記載した書面により招集の請求があったとき。
	\item 監事が第14条第5項第4号の規定に基づいて招集するとき。
\end{enumerate}

\article{総会の招集}
総会は、前条第2項第3号の場合を除いて、理事長が招集する。
\term 理事長は、前条第2項第1号及び第2号の規定による請求があったときは、その日から30日以内に臨時総会を招集しなければならない。
\term 総会を招集するときには、会議の日時、場所、目的及び審議事項を記載した書面又は電磁的方法により、開催の日の少なくとも5日前までに通知しなければならない。

\article{総会の議長}
総会の議長は、その総会に出席した正会員の中から選出する。

\article{総会の定足数}
総会は、正会員総数の2分の1以上の出席がなければ開会することはできない。

\article{総会の議決}
総会における議決事項は、第23条第3項の規定によってあらかじめ通知した事項とする。ただし、緊急の場合については、総会出席者の2分の1以上の同意により議題とすることができる。
\term 総会の議事は、この定款に規定するもののほか、出席した正会員の過半数をもって決し、可否同数のときは、議長の決するところによる。
\term 理事又は正会員が、総会の目的である事項について提案した場合において、正会員全員が書面又は電磁的記録により同意の意思表示をしたときは、当該提案を可決する旨の社員総会の決議があったものとみなす。

\article{総会での表決権等}
各正会員の表決権は平等なものとする。
\term やむを得ない理由により総会に出席できない正会員は、あらかじめ通知された事項について書面若しくは電磁的方法をもって表決し、又は他の正会員を代理人として表決を委任することができる。
\term 前項の規定により表決した正会員は、前2条及び次条第1項の適用については、総会に出席したものとみなす。
\term 総会の議決について、特別の利害関係を有する正会員は、その議事の議決に加わることができない。

\article{総会の議事録}
総会の議事については、次の事項を記載した議事録を作成しなければならない。
\begin{enumerate}
	\item 日時及び場所
	\item 正会員総数及び出席者数(書面若しくは電磁的方法による表決者又は表決委任者がある場合にあっては、その数を付記すること。)
	\item 審議事項
	\item 議事の経過の概要及び議決の結果
	\item 議事録署名人の選任に関する事項
\end{enumerate}
\term 議事録には、議長及び総会において選任された議事録署名人2人が、記名押印又は署名しなければならない。
\term 前2項の規定にかかわらず、正会員全員が書面又は電磁的記録による同意の意思表示をしたことにより、総会の決議があったとみなされた場合においては、次の事項を記載した議事録を作成しなければならない。
\begin{enumerate}
	\item 総会の決議があったものとみなされた事項の内容
	\item 前号の事項の提案をした者の氏名又は名称
	\item 総会の決議があったものとみなされた日及び正会員総数
	\item 議事録の作成に係る職務を行った者の氏名
\end{enumerate}

\article{理事会の構成}
理事会は、理事をもって構成する。

\article{理事会の権能}
理事会は、この定款に別に定める事項のほか、次の事項を議決する。
\begin{enumerate}
	\item 総会に付議すべき事項
	\item 総会の議決した事項の執行に関する事項
	\item その他総会の議決を要しない業務の執行に関する事項
\end{enumerate}

\article{理事会の開催}
理事会は、次に掲げる場合に開催する。
\begin{enumerate}
	\item 理事長が必要と認めたとき。
	\item 理事総数の2分の1以上から理事会の目的である事項を記載した書面により招集の請求があったとき。
\end{enumerate}

\article{理事会の招集}
理事会は、理事長が招集する。
\term 理事長は、前条第2号の規定による請求があったときは、その日から14日以内に理事会を招集しなければならない。
\term 理事会を招集するときは、会議の日時、場所、目的及び審議事項を記載した書面又は電磁的方法により、開催の日の少なくとも5日前までに通知しなければならない。
\label{理事会の通知}

\article{理事会の議長}
理事会の議長は、理事長がこれにあたる。

\article{理事会の議決}
理事会における議決事項は、\ref{理事会の通知}の規定によってあらかじめ通知した事項とする。
\term 理事会の議事は、理事総数の過半数をもって決し、可否同数のときは、議長の決するところによる。

\article{理事会の表決権等}
各理事の表決権は、平等なものとする。
\term やむを得ない理由のため理事会に出席できない理事は、あらかじめ通知された事項について書面をもって表決することができる。
\term 前項の規定により表決した理事は、前条及び次条第1項の適用については、理事会に出席したものとみなす。
\term 理事会の議決について、特別の利害関係を有する理事は、その議事の議決に加わることができない。

\article{理事会の議事録}
理事会の議事については、次の事項を記載した議事録を作成しなければならない。
\begin{enumerate}
	\item 日時及び場所
	\item 理事総数、出席者数及び出席者氏名(書面表決者にあっては、その旨を付記すること。)
	\item 審議事項
	\item 議事の経過の概要及び議決の結果
	\item 議事録署名人の選任に関する事項
\end{enumerate}
\term 議事録には、議長及びその会議において選任された議事録署名人2人が記名押印又は署名しなければならない。

\subsection{資産}
\article{資産の構成}
この法人の資産は、次の各号に掲げるものをもって構成する。
\begin{enumerate}
	\item 設立当初の財産目録に記載された資産
	\item 入会金及び会費
	\item 寄附金品
	\item 財産から生じる収益
	\item 事業に伴う収益
	\item その他の収益
\end{enumerate}

\article{資産の区分}
この法人の資産は、特定非営利活動に係る事業に関する資産とする。

\article{資産の管理}
この法人の資産は、理事長が管理し、その方法は、総会の議決を経て、理事長が別に定める。

\subsection{会計}
\article{会計の原則}
この法人の会計は、法第27条各号に掲げる原則に従って行わなければならない。

\article{会計の区分}
この法人の会計は、特定非営利活動に係る事業会計とする。

\article{事業年度}
この法人の事業年度は、毎年4月1日に始まり、翌年3月31日に終わる。

\article{事業計画及び予算}
この法人の事業計画及びこれに伴う予算は、毎事業年度、理事長が作成し、総会の議決を経なければならない。

\article{暫定予算}
前条の規定にかかわらず、やむを得ない理由により予算が成立しないときは、理事長は、理事会の議決を経て、予算成立の日まで前事業年度の予算に準じ収益費用を講じることができる。
\term 前項の収益費用は、新たに成立した予算の収益費用とみなす。

\article{予算の追加及び更正}
予算成立後にやむを得ない事由が生じたときは、総会の議決を経て、既定予算の追加又は更正をすることができる。

\article{事業報告及び決算}
この法人の事業報告書、活動計算書、貸借対照表及び財産目録等決算に関する書類は、毎事業年度終了後、速やかに、理事長が作成し、監事の監査を受け、総会の議決を経なければならない。
\term 決算上剰余金を生じたときは、次事業年度に繰り越すものとする。

\article{臨機の措置}
予算をもって定めるもののほか、借入金の借入れその他新たな義務の負担をし、又は権利の放棄をしようとするときは、総会の議決を経なければならない。


\subsection{定款の変更、解散及び合併}
\article{定款の変更}
この法人が定款を変更しようとするときは、総会に出席した正会員の4分の3以上の多数による議決を経、かつ、法第25条第3項に規定する事項については、所轄庁の認証を得なければならない。
\term この法人の定款を変更(前項の規定により所轄庁の認証を得なければならない事項を除く。)したときは、所轄庁に届け出なければならない。

\article{解散}
この法人は、次に掲げる事由により解散する。
\begin{enumerate}
	\item 総会の決議
	\item 目的とする特定非営利活動に係る事業の成功の不能
	\item 正会員の欠亡
	\item 合併
	\item 破産手続開始の決定
	\item 所轄庁による設立の認証の取消し
\end{enumerate}
\term 前項第1号の事由によりこの法人が解散するときは、正会員総数の4分の3以上の議決を経なければならない。
\term 第1項第2号の事由によりこの法人が解散するときは、所轄庁の認定を得なければならない。

\article{残余財産の帰属}
この法人が解散(合併又は破産手続開始の決定による解散を除く。)したときに残存する財産は、法第11条第3項に掲げる者のうち、総会において議決したものに譲渡するものとする。

\article{合併}
この法人が合併しようとするときは、総会において正会員総数の4分の3以上の議決を経、かつ、所轄庁の認証を得なければならない。

\subsection{公告の方法}
\article{公告の方法}
この法人の公告は、この法人の掲示場に掲示するとともに、官報に掲載して行う。ただし、法第28条の2第1項に規定する貸借対照表の公告については、この法人のホームページにおいて行う。

\subsection{事務局}
\article{事務局の設置}
この法人に、この法人の事務を処理するため、事務局を設置することができる。
\term 事務局には、事務局長及び必要な職員を置くことができる。

\article{職員の任免}
事務局長及び職員の任免は、理事長が行う。

\article{組織及び運営}
事務局の組織及び運営に関し必要な事項は、総会の議決を経て、理事長が別に定める。

\subsection{雑則}
\article{細則}
この定款の施行について必要な細則は、理事会の議決を経て、理事長がこれを定める。

\subsection*{附則}
\setcounter{TermNo}{0}
\term この定款は、この法人の成立の日から施行する。
\term この法人の設立当初の役員は、別表のとおりとする。
\term この法人の設立当初の役員の任期は、第16条第1項の規定にかかわらず、この法人の成立の日から平成17年6月30日までとする。
\term この法人の設立当初の事業年度は、第43条の規定にかかわらず、この法人の成立の日から平成16年3月31日までとする。
\term この法人の設立当初の事業計画及び収支予算は、第44条の規定にかかわらず、設立総会の定めるところによる。
\term この法人の設立当初の入会金及び会費は、第8条の規定にかかわらず、次に掲げる額とする。
\begin{enumerate}
	\item 入会金(個人・団体) 正会員 10,000円 賛助会員 \phantom{1口} 50,000円
	\item 年会費(個人・団体) 正会員 15,000円 賛助会員 1口 50,000円(一口以上)
\end{enumerate}

\subsection*{附則}
\setcounter{TermNo}{0}
\term この定款は、平成20年6月7日から施行する。
\term この法人の入会金及び会費は、次に掲げる額とする。
\begin{enumerate}
	\item 入会金(個人・団体) 正会員 \phantom{10,00}0円 賛助会員 \phantom{1口 50,00}0円
	\item 年会費(個人・団体) 正会員 \phantom{1}5,000円 賛助会員 1口 50,000円(一口以上)
\end{enumerate}

\subsection*{附則}
\setcounter{TermNo}{0}
\term この定款は、平成27年2月22日から施行する。

\subsection*{附則}
\setcounter{TermNo}{0}
\term この定款は、令和3年2月23日から施行する。

\subsection*{附則}
この定款は、令和5年4月1日から施行する。

\subsection*{附則}
この定款は、令和5年8月1日から施行する。

\subsection*{附則}
\setcounter{TermNo}{0}
\term この定款は、令和6年3月1日から施行する。
\term この法人の入会金及び会費は、第8条の規定にかかわらず、次に掲げる額とする。
\begin{enumerate}
	\item 入会金\hfill 正会員(個人・団体) 0円\hfill 賛助会員(個人・団体) \phantom{1口} 100,000円
	\item 年会費\hfill 正会員(個人・団体) 0円\hfill 賛助会員(個人・団体) 1口 \phantom{1}50,000円\\\hfill(1口以上)
\end{enumerate}

\newpage

\begin{table}[htbp]
	\captionsetup{justification=raggedright,singlelinecheck=false}
	\caption*{\emph{別表} 設立当初の役員}
	\begin{tabular}{|l|l|}
	\hline
	 \kintou{4}{役職名}  &   氏 名    \\ \hline\hline
	 \kintou{4}{理事長}  &  菅 俊雄  \\ \hline
	 副理事長  &  塚本悌三郎  \\ \hline
	 \kintou{4}{理事}  &  楊 秋雄  \\ \hline
	 \kintou{4}{理事}   &  敖 文   \\ \hline
	 \kintou{4}{理事}   &  林 暉   \\ \hline
	 \kintou{4}{監事}   &  大島正克  \\ \hline
	\end{tabular}
\end{table}

\end{document}